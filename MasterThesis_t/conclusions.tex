\chapter{Conclusion}
In this work, we attempted to analyze smart contract and their integration with semantic web data licesing. To achieve this goal, first we tried to analyze blockchain where smart contract reside on it as program code. Afterwards, we showed that how blockchain could be applied as computational paradigm for semantic web and linked data. Then we focused on bitcoin, Etheruem as main usage of blockahin and related vulnerabilities in Ethereum.
Furthermore, we have presented the first initial effort to use this construct in useful ways by extending blockchain with semantic web in supply chain management. As the main goal of this work is to show ssemantic web principle can be use in making smarter smart contract and blockchain network. First, we started presenting our work done in ontology, Ethereum ontology (EthOn) describes blockchain structure and related information such as classes, object properties and etc. Later, we used this concept as RDF triple format which will feed by dataset and will lead into representing our data based on query.