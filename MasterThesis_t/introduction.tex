\section{Introduction}
Smart contract is computer program that expressed the content of agreements and preform transaction on blockchain when specific conditions are met. Smart contract preform verified transaction on blockhain without third party or any supervisors, thus ensuring us to transparent, valid and secure transaction.
As blockain is distributed ledger which store data and transactions, querying them becomes challenges task. This due to the fact that blockchian can allow transaction and payment without needing intermediary.
Moreover there is need to integrate blockchin with semantic web service, thus making use of some linked data tools to index blocks and transaction according to Ethereum ontology.  \\
In this paper, supply chain management is regarded as use case where blokchain is fit for some reasons. 
During product life cycle in every step, data can be documented in blockchain. Blockchain technology can contribute to record single asset as it flows through supply chain node, track orders, payment, product and track digital asset. Blockchain can contribute through distributed nature in sharing information about product process, delivery between from supplier to customers. In today's world, supply chain is complicated structure with multiple involved participants with amount of activities.\\
Security and organizational issue cause to improve the need to build blockchain based supply chain management. In spite of some features of supply chain, blockchain also offers some advantages by indexing, registering products, increase transparency and trust of participants. Besides elimination of third party which allow for growing number of participants, it increases innovation by deploying smart contract with low fee transactions, without cost of third party. \\
The literature provides some supply chain management ontologies for range of activities and industries. Many studies claim the benefit of ontology in supply chain industries which are suggested multiple models to supply chain ontology another application of these ontologies by some Enterprise Modeler program model ontology. This reports different ontologies models in supply chain management fields. Then it works on how semantic web service can be applicable in this field. Existing supply chain ontology has several gaps with respect to domain accuracy, consistency and development approach. But we attempted to semantify blockchain using ontologies and some semantic techniques.Then we purposed proof of concept developed in the concept of supply chain management and semantic blockchain.\\
As this is important to gain advantages, we used Ethereum ontology which provide some specifications of phenomenon, adapt semantic principles for developing semantic blockchain and mapped these concepts using RDF triple and model based on Sparql query to produce blockchain models based on semantic web ontologies.

